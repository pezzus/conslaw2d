\section{Formato dei file}
Vi sono essenzialmente tre formati di file che il programma utilizza: uno riguardante la descrizione della mesh e due per l'output delle soluzioni.

Riguardo la mesh, il file ha una forma di questo tipo:
\begin{verbatim}
# DATA
133     218     54
# POINTS
-1.000000       0.500000
-1.000000       -0.500000
0.000000        -0.200000
[cut]
-0.063683       0.163276
0.163695        -0.167380
-0.069532       -0.241330
# ELEMENTS
3       15      1       58      1
3       22      2       61      0
[cut]
3       104     112     132     1
3       2       126     132     1
# EDGES
0       12      0
12      13      0
13      14      0
[cut]
50      51      13
51      11      13
\end{verbatim}
Il file è suddiviso in quattro sezioni:
\begin{description}
\item[\texttt{\# DATA}] Le tre colonne indicano rispettivamente il numero di vertici, di poligoni e di lati di bordo;
\item[\texttt{\# POINTS}] Ogni riga è un vertice, del quale vengono fornite le coordinate;
\item[\texttt{\# ELEMENTS}] Ogni riga è un poligono, e le colonne sono rispettivamente il numero di lati, l'id dei vertici (in senso antiorario) ed infine il colore dell'elemento;
\item[\texttt{\# EDGES}] Ogni riga è un lato di bordo, del quale si forniscono i vertici e il colore.
\end{description}

Un file di questo tipo può essere facilmente generato in Matlab utilizzando \texttt{matlab2msh.m}, script presente nella directory \texttt{tools}. Da un punto di vista pratico ci si appoggia al toolbox \texttt{pdetool}, col quale genera mesh (il risultato dovrebbero essere tre variabili \texttt{p}, \texttt{e} e \texttt{t}), per poi eseguire:
\begin{verbatim}
>> matlab2msh(p,e,t,'nomefile.msh');
\end{verbatim}
Automaticamente come etichette \texttt{colore} dei poligoni e dei lati vengono assegnate ``d'ufficio'' rispettivamente l'id del sottodominio e l'id del bordo. Attraverso l'interfaccia grafica di \texttt{pdetool} è anche possibile visualizzare queste etichette.

Per quanto riguarda l'output su file della soluzione, come accennato ci sono due formati, il primo per l'output grafico in Matlab mentre il secondo per l'output in Gnuplot.

Nel primo caso si ha semplicemente una lista di valori, dove ogni riga è il valore della soluzione nel vertice corrispondente, secondo lo stesso ordinamento del file della mesh in input. Vi sono inoltre tante colonne quante sono le variabili del problema (4 per Eulero).

La soluzione è automaticamente interpolata sui vertici, e si può visualizzare in Matlab importando il file in una variabile e utilizzando poi i comandi \texttt{pdeplot} o \texttt{trisurf}. Sempre nella directory \texttt{tools} vi sono alcuni esempi pratici, come la generazione di una animazione.

L'ultimo formato riguarda il dataset della soluzione pronto per essere utilizzato in Gnuplot attraverso il comando \texttt{splot}\footnote{Nella visualizzazione tridimensionale potrebbe esserci della grafica corrotta, probabilmente perché il comando è disegnato per utilizzare dataset con griglie cartesiane. Nel nostro caso infatti per rappresentare i triangoli costruiamo dei quadrilateri degeneri.}. In questo caso si può decidere se interpolare la soluzione sui vertici oppure no.